\documentclass[article,A4,11pt]{llncs}%
\usepackage[T1]{fontenc}
\usepackage{amsmath}
\usepackage{amssymb}
\usepackage{epsf,times}
\usepackage{amsfonts}
\usepackage{graphicx}
\usepackage{mathrsfs}
\usepackage{wrapfig}
\usepackage{color}
\usepackage{amsmath,mathrsfs,bm}
\usepackage{cases}
\usepackage{subfig}

\leftmargin=0.2cm
\oddsidemargin=1.2cm
\evensidemargin=0cm
\topmargin=0cm
\textwidth=15.5cm
\textheight=21.5cm
\pagestyle{plain}
\begin{document}

% *****************************
% *   YOUR TEXT STARTS HERE   *
% *****************************

\title{Title of Your Talk (Note Capitalization)}
\author{} \institute{} % Intentionally left blank
\tocauthor{First Author, \underline{Second Author}}
\maketitle
\begin{center}
{\large First Author}\\
Address of First Author\\
{\tt email@of.first.author}\\
\vspace{4mm} % Use this space when including 3rd author
{\large \underline{Second Author}}\\
Address of Second Author\\
{\tt email@of.second.author}
\end{center}

\section*{Abstract}

Enter your abstract here. Authors of contributed lectures: Please do not 
exceed one page including references. References to related or 
competitive work are mandatory. Presenting author should be underlined. 
Please do not alter the internal structure of the template. Do not 
introduce any new definitions or commands, they cause problems during the 
compilation of the final Book of Abstract. The Book of Abstracts will be 
compiled using pdflatex. If using images, please make sure thay are
in PDF or PNG. Thank you!


\bibliographystyle{plain}
\begin{thebibliography}{10}

\bibitem{CockburnGopalakrishnan04}
{\sc B.~Cockburn and J.~Gopalakrishnan}. {A characterization of hybridized
  mixed methods for second order elliptic problems}. SIAM J. Numer. Anal. 42
  (2004), pp.~283--301.

\bibitem{EwingWangYang03}
{\sc R.~Ewing, J.~Wang, and Y.~Yang}. {A stabilized discontinuous finite
  element method for elliptic problems}. Numer. Linear Alg. Appl. 10 (2003),
  pp.~83--104.

\bibitem{A104}
{\sc Randolph~E. Bank, Jinchao Xu, Bin Zheng}.
\newblock Superconvergent derivative recovery for {Lagrange} triangular
  elements of degree $p$ on unstructured grids.
\newblock SIAM J.~Numer. Anal. 45 (2007), pp. 2032--2046. 

\end{thebibliography}

% ***************************
% *   YOUR TEXT ENDS HERE   *
% ***************************

\end{document}
